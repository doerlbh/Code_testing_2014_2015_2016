\documentclass[fleqn,10pt]{wlscirep}
\title{Stochastic Dynamics of RNA interference: A Single-Molecule Time-based Approach}

\author[1*]{Baihan Lin}
\affil[1]{Department of Applied Mathematics, University of Washington, Seattle, WA 98195, USA}
\affil[*]{doerlbh@gmail.com}

%\keywords{Keyword1, Keyword2, Keyword3}

\begin{abstract}
Example Abstract. Abstract must be under 200 words and not include subheadings or citations. Example Abstract. Abstract must be under 200 words and not include subheadings or citations. Example Abstract. Abstract must be under 200 words and not include subheadings or citations. Example Abstract. Abstract must be under 200 words and not include subheadings or citations. Example Abstract. Abstract must be under 200 words and not include subheadings or citations. Example Abstract. Abstract must be under 200 words and not include subheadings or citations. Example Abstract. Abstract must be under 200 words and not include subheadings or citations. Example Abstract. Abstract must be under 200 words and not include subheadings or citations. 

Under a general mechanistic understanding of RNA interference, we present a stochastic model for the cellular dynamics of gene expression involved in RNA interference in a single-molecule time-based approach.

 

\end{abstract}
\begin{document}

\flushbottom
\maketitle
\thispagestyle{empty}

\noindent Please note: Abbreviations should be introduced at the first mention in the main text – no abbreviations lists. Suggested structure of main text (not enforced) is provided below.

\section*{Introduction}

The Introduction section, of referenced text\cite{Figueredo:2009dg} expands on the background of the work (some overlap with the Abstract is acceptable). The introduction should not include subheadings.

\section*{Results}

To simply the model, I used iRNA to represent microRNA and siRNA, which are similar RNA interference agents. Below is the proposed model on gene expression. Instead of producing transcription factor to bind with DNA, iRNA tends to bind with mRNA and form a complex to inhibit the translation. I made several assumptions, including same degradation rate for iRNA and mRNA, irreversible inhibition f = 0, etc..
 
Here is what I got so far (attached images 1~4):
 
As you can see from my calculation (page 4), the transition diagram is in 3D instead of 2D (as Qian 2010), and I am having trouble figuring out the initial steady state distribution to put into the PGF. Apparently, mRNA and iRNA doesn’t fit Poisson distribution any more as they are intercorrelated instead of simply happens by chance sporadically.




\subsection*{Subsection}

\begin{equation}
\ x=\frac{-b\pm\sqrt{b^2-4ac}}{2a}
\end{equation}




Example text under a subsection. Bulleted lists may be used where appropriate, e.g.

\begin{itemize}
\item First item
\item Second item
\end{itemize}

\subsubsection*{Third-level section}
 
Topical subheadings are allowed.

\section*{Discussion}

The Discussion should be succinct and must not contain subheadings.

\section*{Methods}

Topical subheadings are allowed. Authors must ensure that their Methods section includes adequate experimental and characterization data necessary for others in the field to reproduce their work.

\bibliography{sample}

\noindent LaTeX formats citations and references automatically using the bibliography records in your .bib file, which you can edit via the project menu. Use the cite command for an inline citation, e.g.  \cite{Figueredo:2009dg}.

\section*{Acknowledgements}

Thank Prof. Qian for giving us insightful lectures about the mathematical theories of cellular dynamic and the exciting field of mathematical biology. Thank him for always setting challenging questions for me to explore! Thank my friends in AMATH 531 who come along with me in great energy! Thank University of Washington for giving me the platform to scientifically explore problems and subjects! I will continue the voyage of exploring the infinite realm of mathematical and systems biology in my academic career fearlessly.

\section*{Additional information}

For supporting MATLAB Codes, please refer to Supplementary Information Attached. All code for the reproduction of the reported results can be downloaded from my \href{https://github.com/doerlbh/RNAi_CME_Dynamics}{GitHub Repository}.

\begin{figure}[ht]
\centering
\includegraphics[width=\linewidth]{stream}
\caption{Legend (350 words max). Example legend text.}
\label{fig:stream}
\end{figure}

\begin{table}[ht]
\centering
\begin{tabular}{|l|l|l|}
\hline
Condition & n & p \\
\hline
A & 5 & 0.1 \\
\hline
B & 10 & 0.01 \\
\hline
\end{tabular}
\caption{\label{tab:example}Legend (350 words max). Example legend text.}
\end{table}

Figures and tables can be referenced in LaTeX using the ref command, e.g. Figure \ref{fig:stream} and Table \ref{tab:example}.

\end{document}